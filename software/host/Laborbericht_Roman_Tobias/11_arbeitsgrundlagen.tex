In diesem Kapitel werden alle Grundlagen erläutert, welche nötig sind um den Versuch durchführen zu können. 

\subsection{Die Lichtgeschwindkeit}
\label{Die Lichtgeschwindkeit}

An der Internationalen Konferenz für Mass und Gewicht kurz ICPM wurde der Meter durch Festlegung des Zahlenwertes für die Lichgeschwindigkeit im Vakuum wie folgt neu definiert.

%%%%%%%%%%%%%%%%%%%%%%%%%%%%%%%%%%%%%%%%%%%%%%%%%%%%%%%%%%%%%%%%%%%%%%%%%%%%%
\begin{equation*}
c_{0} = 299'792'458 m/s
\label{eq:Lichtgeschwindigkeit}
\end{equation*}
%%%%%%%%%%%%%%%%%%%%%%%%%%%%%%%%%%%%%%%%%%%%%%%%%%%%%%%%%%%%%%%%%%%%%%%%%%%%%

Die Lichtgeschwindigkeit in einem Medium wiederrum erhält man mit Hilfe des berechneten Brechungsindexes $n$ definiert in Formel \ref{eq:Formel_Brechungsindex}. In einem Gas, also zum Beispiel in der Luft, ist $n$ proportional zur Dichte und wird unteranderem noch durch die Luftfeuchtigkeit leicht beeinflusst.

%%%%%%%%%%%%%%%%%%%%%%%%%%%%%%%%%%%%%%%%%%%%%%%%%%%%%%%%%%%%%%%%%%%%%%%%%%%%%
\begin{equation}
c=c_{0}/n
\label{eq:Formel_Brechungsindex}
\end{equation}
%%%%%%%%%%%%%%%%%%%%%%%%%%%%%%%%%%%%%%%%%%%%%%%%%%%%%%%%%%%%%%%%%%%%%%%%%%%%%

Um den Brechungsindex $n$ auszurechnen, kann Formel \ref{eq:Formel_Brechungsindex_in_der_Luft} angewandt werden, wobei diese jedoch noch nach $n$ umgeformt werden muss.

%%%%%%%%%%%%%%%%%%%%%%%%%%%%%%%%%%%%%%%%%%%%%%%%%%%%%%%%%%%%%%%%%%%%%%%%%%%%%
\begin{equation}
(n - 1) = (n_{n} - 1)\cdot\dfrac{p \cdot T_{n}}{p_{n} \cdot T}-(\beta - \dfrac{\gamma}{\lambda}_{0}) \cdot p_{w}
\label{eq:Formel_Brechungsindex_in_der_Luft}
\end{equation}
%%%%%%%%%%%%%%%%%%%%%%%%%%%%%%%%%%%%%%%%%%%%%%%%%%%%%%%%%%%%%%%%%%%%%%%%%%%%%

Die in Formel \ref{eq:Formel_Brechungsindex_in_der_Luft} enthaltenen Parameter sind unteranderem die gemessene Raumtemperatur $T$ in Kelvin, sowie der gemessene Luftdruck $p$ in Bar. $\beta$ und $\gamma$ sind Wasserdampfkorrekturfaktoren, sowie $\lambda_{0}$ die Vakuum-Wellenlänge, welche unmerklich verschieden von der Luftwellenlänge ist, weshalb diese verwendet wird.

%%%%%%%%%%%%%%%%%%%%%%%%%%%%%%%%%%%%%%%%%%%%%%%%%%%%%%%%%%%%%%%%%%%%%%%%%%%%%
\begin{equation*}
\beta = 4.292 \cdot 10^{-8} mbar^{-1}
\label{eq:Faktor_Beta}
\end{equation*}
\begin{equation*}
\gamma = 3.43 \cdot 10^{-2} (nm)^{2}mbar^{-1}
\label{eq:Faktor_Gamma}
\end{equation*}
\begin{equation*}
\lambda_{0} = 632.8 nm
\label{eq:Faktor_Lambda}
\end{equation*}
%%%%%%%%%%%%%%%%%%%%%%%%%%%%%%%%%%%%%%%%%%%%%%%%%%%%%%%%%%%%%%%%%%%%%%%%%%%%%

Mit Hilfe der nachfolgenden Abbildung \ref{fig:Wasserdampfsättigung und Normbrechungsindex}, können die restlichen Parameter bestimmt werden. So kann beispielsweise anhand der gemessenen Raumtemperatur $T$ aus der linken Abbildung die p-Sättigung oder in der rechten Abbildung auf Grund der vorgegebenen Wellenlänge der Normbrechnungsindex $(n_{n} - 1)$ herausgelesen werden. Die restlichen Parameter $T_{n}$ und $p_{n}$ sind ebenfalls in der rechten Abbildung aus dem Text ablesbar und stellen die Parameter dar, auf welche die Abbildung rechts normiert ist.

%%%%%%%%%%%%%%%%%%%%%%%%%%%%%%%%%%%%%%%%%%%%%%%%%%%%%%%%%%%%%%%%%%%%%%%%%%%%%
\begin{figure}[b]
\includegraphics[width=\textwidth]{Brechungsindex_Luft_Grafik.png}
\caption{Wasserdampfsättigung und Normbrechungsindex}
\label{fig:Wasserdampfsättigung und Normbrechungsindex}
\end{figure}
%%%%%%%%%%%%%%%%%%%%%%%%%%%%%%%%%%%%%%%%%%%%%%%%%%%%%%%%%%%%%%%%%%%%%%%%%%%%%

Der Parameter $p_{w}$ wiederrum ist der Wasserdampfpartialdruck und wird separat mit Formel \ref{eq:Formel_Wasserdampfpartialdruck} ausgerechnet, wobei der aus Abbildung \ref{fig:Wasserdampfsättigung und Normbrechungsindex} gelesene Wert für die p-Sättigung verwendet wird, sowie die gemessene relative Luftfeuchtigkeit $rF$ in Prozent während der Versuchsdurchführung.

%%%%%%%%%%%%%%%%%%%%%%%%%%%%%%%%%%%%%%%%%%%%%%%%%%%%%%%%%%%%%%%%%%%%%%%%%%%%%
\begin{equation}
p_{w} = \dfrac{rF}{100\%} \cdot p-Saettigung
\label{eq:Formel_Wasserdampfpartialdruck}
\end{equation}
%%%%%%%%%%%%%%%%%%%%%%%%%%%%%%%%%%%%%%%%%%%%%%%%%%%%%%%%%%%%%%%%%%%%%%%%%%%%%

Werden die aus Abbildung \ref{fig:Wasserdampfsättigung und Normbrechungsindex} herausgelesenen Werte und der mit Formel \ref{eq:Formel_Wasserdampfpartialdruck} berechnete Wert in Formel \ref{eq:Formel_Brechungsindex_in_der_Luft} eingesetzt sowie nach $n$ aufgelöst, erhält man der spezifische Brechungsindex $n$ für die gemessenen Versuchsumgebungseinflüsse. Dieser ermittelte Brechungsindex $n$, kann dann wiederrum in Formel \ref{eq:Formel_Brechungsindex} eingesetzt werden und man erhält den für die Versuchsumgebung spezifischer Wert der Lichtgeschwindigkeit $c$ in der Luft. 

\subsection{Messung der Lichtgeschwindigkeit nach Michelson}

Zur Bestimmung der Lichtgeschwindigkeit gibt es verschiendeste Methoden. Eine davon ist die Messung der Lichtgeschwindigkeit nach Michelson, welche die Funktion der Abbildung und Kollimation in einer langbrennweitigen Linse vereint. Dadurch kann erreicht werden, dass die Abbildungsqualität verbessert wird und man hat die Möglichkeit einer gewissen Freiheit bei der Wahl der Basisstrecke.\\[4mm]
Um den Strahlengang bei der Michelson Methode verstehen zu können, wird die Abbildungsgleichung sowie der Abbildungsmasstab $\beta$ benötigt.

%%%%%%%%%%%%%%%%%%%%%%%%%%%%%%%%%%%%%%%%%%%%%%%%%%%%%%%%%%%%%%%%%%%%%%%%%%%%%
\begin{equation}
\dfrac{1}{f} = \dfrac{1}{g} + \dfrac{1}{b}
\label{eq:Formel_Abbilungsgleichung}
\end{equation}
\begin{equation}
\beta = \frac{B}{G} = \frac{b}{g}
\label{eq:Formel_Abbildungsmassstab}
\end{equation}
%%%%%%%%%%%%%%%%%%%%%%%%%%%%%%%%%%%%%%%%%%%%%%%%%%%%%%%%%%%%%%%%%%%%%%%%%%%%%

In den Formeln \ref{eq:Formel_Abbilungsgleichung} und \ref{eq:Formel_Abbildungsmassstab} enthalten sind $f$ die Brennweite, $g$ die Gegenstandsweite, $b$ die Bildweite, $B$ die Bildgrösse und $G$ die Gegenstandsgrösse. Abbildung \ref{fig:Abbildung an einer Linse} zeigt die Anwendung dieser Formeln anhand einer Linse.

%%%%%%%%%%%%%%%%%%%%%%%%%%%%%%%%%%%%%%%%%%%%%%%%%%%%%%%%%%%%%%%%%%%%%%%%%%%%%
\begin{figure}[htb]
\includegraphics[width=\textwidth]{Abbildung_Linse_Grafik.png}
\caption{Abbildung an einer Linse}
\label{fig:Abbildung an einer Linse}
\end{figure}
%%%%%%%%%%%%%%%%%%%%%%%%%%%%%%%%%%%%%%%%%%%%%%%%%%%%%%%%%%%%%%%%%%%%%%%%%%%%%

Abbildung \ref{fig:Strahlengang Michelson} wiederrum zeigt den Strahlengang. Als Messmarke dient anstatt eines Gitters ein einstellbarer vertikaler Spalt. Das an diesem Spalt leicht divergent auslaufende Licht wird durch die Linse $L_{1}$ abgebildet, wobei die Distanz zwischen Spalt und $L_{1}$ möglichst genau auf die Brennweite $f_{1}$ eingestellt werden muss. Der Drehspiegel DS wiederrum soll so justiert werden, dass das Licht auf die Mitte des Hohlspiegels HS trifft. Dieser Hohlspiegel HS bildet den Spalt in seine Brennebene ab, wo sich der Endspiegel ES befindet. Mit Hilfe des Strahlenteilers wird sodann das rückreflektierte Bild im Messokular sichtbar.

Bei einem sich rotierenden Spiegel trifft das vom Endspiegel ES zurückkehrende Licht den Drehspiegel um einen Winkel $\delta$ gedreht. Daher gilt nachfolgende Formel \ref{eq:Formel_Winkeländerung}.

%%%%%%%%%%%%%%%%%%%%%%%%%%%%%%%%%%%%%%%%%%%%%%%%%%%%%%%%%%%%%%%%%%%%%%%%%%%%%
\begin{equation}
\delta = \omega\cdot\Delta t = \omega\cdot\dfrac{2\cdot(s_{2}+f_{2})}{c}
\label{eq:Formel_Winkeländerung}
\end{equation}
%%%%%%%%%%%%%%%%%%%%%%%%%%%%%%%%%%%%%%%%%%%%%%%%%%%%%%%%%%%%%%%%%%%%%%%%%%%%%

%%%%%%%%%%%%%%%%%%%%%%%%%%%%%%%%%%%%%%%%%%%%%%%%%%%%%%%%%%%%%%%%%%%%%%%%%%%%%
\begin{figure}[htb]
\includegraphics[width=\textwidth]{Strahlengang_Michelson_Grafik.png}
\caption{Strahlengang Michelson}
\label{fig:Strahlengang Michelson}
\end{figure}
%%%%%%%%%%%%%%%%%%%%%%%%%%%%%%%%%%%%%%%%%%%%%%%%%%%%%%%%%%%%%%%%%%%%%%%%%%%%%

In Formel \ref{eq:Formel_Winkeländerung} enthalten ist die Kreisfrequenz $\omega$ des Drehspiegels und $\Delta t$ die Laufzeit des Lichtes vom Drehspiegel zum Endspiegel und zurück. Genau diese Drehung des Spiegels führt zu einer Richtungsänderung der Bündelachse um den Winkel $2\delta$ und damit auch zu einer seitlichen Verschiebung x wie in Abbildung \ref{fig:Strahlengang Michelson} ersichtlich ist.\\
Durch die mit Formel \ref{eq:Formel_Kleinwinkelnäherung_Michelson} angewandte Kleinwinkelnäherung ergibt sich schliesslich die gesuchte Formel \ref{eq:Formel_Lichtgeschwindigkeit_Michelson} zur Bestimmung der Lichgeschwindigkeit $c$ nach der Michelson Methode.

%%%%%%%%%%%%%%%%%%%%%%%%%%%%%%%%%%%%%%%%%%%%%%%%%%%%%%%%%%%%%%%%%%%%%%%%%%%%%
\begin{equation}
2\cdot\delta = x/f_{1}
\label{eq:Formel_Kleinwinkelnäherung_Michelson}
\end{equation}
%%%%%%%%%%%%%%%%%%%%%%%%%%%%%%%%%%%%%%%%%%%%%%%%%%%%%%%%%%%%%%%%%%%%%%%%%%%%%

%%%%%%%%%%%%%%%%%%%%%%%%%%%%%%%%%%%%%%%%%%%%%%%%%%%%%%%%%%%%%%%%%%%%%%%%%%%%%
\begin{equation}
c = 4\cdot\omega\cdot\dfrac{(s_{2}+f_{2})\cdot f_{1}}{x}
\label{eq:Formel_Lichtgeschwindigkeit_Michelson}
\end{equation}
%%%%%%%%%%%%%%%%%%%%%%%%%%%%%%%%%%%%%%%%%%%%%%%%%%%%%%%%%%%%%%%%%%%%%%%%%%%%%