Nachfolgend eine Auflistung und Analyse der Resultate aus der Auswertung.\\

\subsection{Beugung am Spalt}
Es wird der Berechnete Spaltabstand verglichen mit dem angegebenen Spaltabstand des Objektes.\\
\\
\textbf{Durch Beobachten :}
\begin{tabbing}
\hspace{40mm}			\=  	\\
Gemessen: \> $\underline{86\pm 2\mu m}$			\\
Soll:	\> $\underline{100\pm 5 \mu m}$			\\
\\
Gemessen: \> $\underline{25.6 \pm 0.8\mu m}$		\\
Soll:	  \> $\underline{30 \pm 5\mu m}$			\\
\\
\textbf{Mit Linse:}		\\
\\
Gemessen: \> $\underline{86\pm 3\mu m}$			\\
Soll:	\> $\underline{100\pm 5 \mu m}$			\\
\\
Gemessen: \> $\underline{25.2 \pm 0.8\mu m}$		\\
Soll:	\> $\underline{30 \pm 5\mu m}$			\\
\end{tabbing}

\subsection{Beugung am Loch}
Es wird der Berechnete Lochabstand verglichen mit dem angegebenen Lochabstand des Objektes.\\
\\
\textbf{Durch Beobachten:}
\begin{tabbing}
\hspace{40mm}			\=  	\\
Gemessen: \> $\underline{121 \pm 7\mu m}$			\\
Soll:	\> $\underline{150 \pm 5 \mu m}$			\\
\\
Gemessen: \> $\underline{85 \pm 4\mu m}$		\\
Soll:	\> $\underline{100 \pm 5\mu m}$			\\
\\
\textbf{Mit Linse:}		\\
\\
Gemessen: \> $\underline{123 \pm 7\mu m}$			\\
Soll:	\> $\underline{150\pm 5 \mu m}$			\\
\\
Gemessen: \> $\underline{83 \pm 5\mu m}$		\\
Soll:	\> $\underline{100 \pm 5\mu m}$			\\
\end{tabbing}

\subsection{Beugung am Gitter}
Es wird der Berechnete Gitterabstand verglichen mit dem angegebenen Gitterabstand des Objektes.\\
\\
\textbf{Durch Beobachten:}
\begin{tabbing}
\hspace{40mm}			\=  	\\
Gemessen: \> $\underline{12.57 \pm 0.2\mu m}$			\\
Soll:	\> $\underline{12.5\pm 5 \mu m}$			\\
\\
Gemessen: \> $\underline{9.4 \pm 0.5\mu m}$		\\
Soll:	\> $\underline{10 \pm 5\mu m}$			\\
\\
\textbf{Mit Linse:}		\\
\\
Gemessen: \> $\underline{12.64 \pm 0.2\mu m}$			\\
Soll:	\> $\underline{12.5\pm 5 \mu m}$			\\
\\
Gemessen: \> $\underline{10.3 \pm 0.5\mu m}$		\\
Soll:	\> $\underline{10 \pm 5\mu m}$			\\
\end{tabbing}

Es ist ersichtlich, dass die Messungen der Gitter die höchste Genauigkeit aufweisen. Dies liegt daran, dass bei den Gittern der Abstand zur Matscheibe sehr klein gewählt werden konnte. Dies ermöglichte eine genauere Vermessung des Abstandes zum Objekt. Des weiteren waren die Maximas viel deutlicher Abzulesen als bei den anderen Objekten.

\subsection{Schlusswort}
Bei diesem Versuch habe ich einige interessante und neue Sachen gelernt. Die Theorie war anspruchsfoll wie auch der Versuchsaufbau, da es nicht einfach wahr eine brauchbare Messung zu erzeugen. Es hat sich gezeigt, dass die Länge zum Objekt möglichst klein gewählt werden sollte oder dann mit einem anderen Messmittel, wie z.B einem Lasermessgerät ermittel werden sollte. Versuch war alles in allem interresant und lehrreich.

\vspace{3mm}
Datum:\hspace{50mm}Unterschrift:
