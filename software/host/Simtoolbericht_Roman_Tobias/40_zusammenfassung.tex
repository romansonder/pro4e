Es war erstaundlich wie ausgeklügelt die einzelnen Elemente der Simulation programmiert sind. Wir waren zuerst unsicher über die Funktionalität der Simulation. Umso mehr erfreute es uns zu sehen, wie die erwünschten Ergebnisse eintrafen. Die Simulation war für unsere Zwecke ausreichend Komplex und wir konnten unterschiedliche physikalische Eigenschaften verifizieren. Dieser Versuch zeigte uns auf, mit welchen eifachen Mitteln eine komplexe Aufgabenstellung gelöst werden kann. Wir würden nun auch in weiteren Aufgabenstellungen auf eine Simulation zurückgreifen.\\

Es war eine positive Erfahrung für uns, auch wenn der erste Kontakt mit dem Simulationstool sich schwieriger gestaltete als erwartet. Die Einarbeitung in die Elemente, welche für die hydraulischen und flüssigen Eigenschaften der Wasserpumpe benötigt wurden, beanspruchte den grössten Teil der Zeit, welche wir für diese Arbeit benötigten. Danach war es jedoch einfach, die erwünschte Konfiguration von Simulationstool-Blöcken zu erreichen.\\

Wir haben viel von diesem Versuch gelernt und empfehlen ihn gerne an weitere Klassen weiter. Es ist eine gute Möglichkeit sich mit dem gelernten auseinander zu setzen. Auch ermöglicht dieser Versuch es einer eigenen Idee nachzugehen und sich von den Möglichkeiten dieses Simultiontool zu überzeugen.
