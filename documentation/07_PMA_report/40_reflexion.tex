Im  Folgenden werden jeweils zwei positive und zwei negative Projekterfahrungen reflektiert und die daraus gewonnenen Erkenntnisse erläutert.

\subsection{Die Analysephase des Projektes wurde schnell abgeschlossen}

Voller Tatendrang und motiviert durch den Wunsch möglichst bald mit dem Bau des Prototyps starten zu können, stürzten wir uns sogleich in den ersten beiden Semesterwochen auf die Analyse der Problemstellung. Die Arbeit wurde gleich am ersten Tag auf alle Teammitglieder aufgeteilt und alle nutzten die eher moderat ausgelastete Zeit, um sich in ihr Teilgebiet einzuarbeiten und die benötigten Informationen zusammenzutragen.\\
Die ersten Ergebnisse dieses frühen Efforts liessen sich bereits in der zweiten Semesterwoche zeigen und wir wussten bereits bis auf ein paar Details, wie wir diese Herausforderung meistern können. Die Analyse konnte dann sogar bis in der dritten Semesterwoche nahezu abgeschlossen werden.\\
\\
Ausschlaggebend für diese Entwicklung war bestimmt, dass sich viele unserer Teammitglieder auf den Bau des Prototypen freuten. Genau diese Freude steckte die Anderen an und sie setzten sich für ein zeitnahes Resultat ein. Ein schönes Sprichwort, das dieses Phänomen umschreibt, lautet:  ''Wenn Du ein Schiff bauen willst, dann trommle nicht Männer zusammen um Holz zu beschaffen, Aufgaben zu vergeben und die Arbeit einzuteilen, sondern lehre die Männer die Sehnsucht nach dem weiten, endlosen Meer.'' \cite{SCHIFF} und genau das geschah.\\
\\
Diesem frühen Einsatz des ganzen Teams hatten wir es zu verdanken, dass wir sehr schnell einen ersten Prototypen herstellen konnten. Es stellte sich heraus, dass dieser frühe Erfolg uns durch die gesamte Projektzeit zugutekam. Nun behalten wir diese positive Entwicklung im Hinterkopf für weitere Projekte.

\subsection{Die Teammitglieder erledigten ihre Arbeiten seriös und allfällige Verzögerungen wurden rechtzeitig gemeldet}

Nachdem die Analysephase abgeschlossen war, wurden die Aufgaben an die einzelnen Teammitglieder verteilt. 
Dadurch, dass viele Teammitglieder den Prototypen schnell erledigt haben wollten, wurden die Arbeiten schnell und seriös erledigt. Durch das seriöse Erledigen der Arbeiten benötigte es fast keine Korrekturen, sodass die erledigten Arbeiten nicht mehrmals erledigt werden mussten.\\
\\
Das seriöse Arbeiten der einzelnen Teammitglieder war nicht nur vorbildlich, sondern förderte auch die Hilfsbereitschaft innerhalb des Teams. So konnten selbst unangenehmere Arbeiten verteilt und erledigt werden. Dies funktionierte auch, als einzelne Mitglieder durch Krankheiten kurzfristig ausfielen. Selbst solche Absenzen konnten durch die Hilfsbereitschaft innerhalb des Teams kompensiert werden. So gab es nie negative Auswirkungen auf den Erfolg des Projekts.\\
\\
Die Seriosität in Zusammenarbeit mit der Hilfsbreitschaft funktionierte nur dank der erfolgreichen Kommunikation innerhalb des Teams. Es wurde mehrere Möglichkeiten Informationen weiterzuleiten angeboten. Für zukünftige Projekte ist genau so eine Einstellung erwünscht. 

\subsection{Die Softwarearbeit wurde unterschätzt}

Bereits schon in der Analysephase wurde der Fehler gemacht, dass der Fokus grösstenteils auf die Analyse der Hardware gesetzt wurde. Dies vor allem aus dem Grund, da die Software auf der Hardware basiert und wir so früh wie möglich einen ersten Prototypen realisieren wollten. Voller Elan für die Hardware wurde jedoch die Analyse in Sachen Softwarearbeiten völlig unterschätzt.\\
\\
Ein weiterer Grund für diese Fehlplanung war, dass wir aufgrund der vom Hersteller des Mikrokontrollers zur freien Verfügung gestellten Bibliotheken davon ausgingen, hierfür nicht viel Zeit investieren zu müssen. Wie sich dann aber schnell herausstellte, war lediglich das Einarbeiten in diese Entwicklungsumgebung und Bibliotheken schon sehr zeitintensiv.\\
\\
Gegen Ende des Projektes zeigten sich dann immer mehr Tücken im Code, welche von der hauptzuständigen Person nicht alleine gelöst werden konnten. Beim gemeinsamen Lösen der Problemstellen zeigte sich dann aber schnell, dass diese nicht aufgrund von fehlendem Know-How waren, sondern vielmehr ''Timing-Issues'' waren, welche schwer vorhersehbar sind.\\
\\
Im Hinterkopf behalten wir deshalb, dass das Vieraugenprinzip schnell zu Lösungen führt und die Softwarearbeiten auch bei bereits vorhandenen Bibliotheken keinesfalls unterschätzt werden dürfen.


\subsection{Es funktionieren nicht alle SD-Kartentypen}
Bei neuen Ausstellungen müssen die Sounddateien auf dem Dojo ausgewechselt werden. Dabei wird er über ein USB-Kabel mit einem Computer verbunden und die Daten werden auf einer Speicherkarte abgespeichert. \\
\\
Um die Verbindung zwischen USB und der Speicherkarte zu realisieren, standen zwei Möglichkeiten zur Verfügung. Einerseits hätte man einen Mikrokontroller mit USB-fähiger Peripherie verwenden können und andererseits konnte ein Baustein ausfindig gemacht werden, welcher diese Aufgabe selbstständig übernehmen kann. Um den Softwareaufwand niedriger zu halten, entschied sich das Team für Letzteres. Der verwendete Baustein ist der VUB300 von Elan Microelectronics. Es handelt sich hier um einen etwas exotischen Baustein, der als USB/SDIO-Bridge eingesetzt werden kann. Man findet dazu relativ wenige Informationen. Leider funktioniert bei diesem der Zugriff auf Speicherkarten nur bei bestimmten Karten. Welche Karten sich eigenen ist unklar, da kein Muster betreffend Format, Grösse oder Alter erkannt werden konnte.  Schliesslich konnte nur eine Speicherkarte gefunden werden, welche sich für unsere Anwendung eignet. Somit traf das Risiko USB Interface teilweise ein, da die USB-Verbindung nur bei einem Endprodukt realisiert werden konnte.  \\
\\
Durch das Ereignis entstanden rund 7 Arbeitsstunden Mehraufwand.  Um das Problem sauber zu beheben, wäre eine fundamentale Änderung im Hardwarekonzept nötig, was zum Zeitpunkt als man das Problem entdeckte, nicht mehr möglich war. Da es sich beim Projekt um eine Machbarkeitsstudie handelt und der Zugriff auf die Speicherkarte nicht essenziell für die Funktionsweise des Dojos ist, war es trotz des beschriebenen Problems möglich das Projekt erfolgreich zu beenden.  Einzige Folge ist, dass bei zwei der drei produzierten Prototypen die USB-Verbindung nicht funktioniert.\\
\\
Als Workaround können die SD-Karten entnommen und direkt im Computer beschrieben werden. Aufgrund des hier beschriebenen Problems wird bei einer allfälligen Weiterentwicklung des Produktes stark davon abgeraten, den VUB300 weiterhin zu verwenden. Stattdessen wird empfohlen, den Nachfolger von verwendetem Mikrocontroller zu verwenden. Dieser (nRF52840) bietet eine USB-Peripherie an und ist fähig auf eine Speicherkarte zuzugreifen. \\
\\
Um in künftigen Projekten nicht auf ähnliche Probleme zu stossen, müssen allfällige exotische Chips auf jeden Fall gut evaluiert werden. Falls dafür nicht genügend Informationen zur Verfügung stehen, sollte falls möglich auf eine Implementation verzichtet werden. 