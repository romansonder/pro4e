Im  Folgenden werden jeweils zwei positive und zwei negative Projekterfahrungen reflektiert und die daraus gewonnenen Erkenntnisse erläutert.

\subsection{Die Analysephase des Projektes wurde schnell abgeschlossen}

Foller Tatendrang und motiviert durch den Wunsch möglichst bald mit dem Prottybenbau starten zu können, stürtzten wir uns sogleich in den ersten beiden Semesterwochen auf die Analyse der Problemstellung. Die Arbeit wurde gleich am ersten Tag auf alle Teammitglieder aufgeteilt und alle nutzten die eher moderat ausgelastet Zeit um sich in Ihr Teilgebiet einzuarbeiten und die benötigten Informationen zusammen zu tragen.\\
Die ersten Ergebinisse dieses frühen Efforts liessen sich bereits in der zweiten Semesterwoche zeigen und wir wussten bereits bis auf ein paar Details wie wir dise Herausforderung meistern können. Die Analyse konnte dann sogar bis in der driten Semesterwoche nahezu abgeschlossen werden.\\
\\
Ausschlaggebend für diese Entwicklung war bestimmt, dass sich in unserem Team viele auf den Bau des Prototypen freuten und genau diese Freude steckte die Andere an und sie setzten sich für ein zeitnahes Resultat ein. Ein Schönes Sprichwort das dieses Phenomen umschreibt lautet:  ''Wenn Du ein Schiff bauen willst, dann trommle nicht Männer zusammen um Holz zu beschaffen, Aufgaben zu vergeben und die Arbeit einzuteilen, sondern lehre die Männer die Sehnsucht nach dem weiten, endlosen Meer.'' \cite{SCHIFF} und genau das geschah.\\
\\
Diesem frühen Einsatz des ganzen Teams hatten wir es zu Verdanken, dass wir sehr schnell einen ersten Prototypen herstellen konnten. Es stellte sich heraus, dass dieser frühe Erfolg uns durch die gesamte Projektzeit zu gute kam. Wir behalten diese positive Entwicklum im Hinterkopf für weitere Projekte.
