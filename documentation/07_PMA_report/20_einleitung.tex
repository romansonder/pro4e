Um das Erlebnis eines Museumsbesuchs zu verbessern, sollten Besucher nicht nur in der Lage sein Ausstellungsstücke anzusehen. Vielmehr sollte es möglich sein, spezifische Informationen auditiv über das Kunstobjekt zu erhalten.\\\\
Ziel des Projektes ist es, einen interaktiven Museumsguide für Museumsbesucher zu entwickeln. Das handliche Gerät namens Dojo wurde funktionell und gestalterisch von Jana Kalbermatter entwickelt. In diesem Projekt erfolgt die Realisierung der Technik dahinter. \\
Dojo ist ein portables, stabförmiges Gerät zur Lokalisierung von Kunstobjekten. Es erkennt umliegende Ausstellungsstücke anhand sogenannter Beacons und benachrichtigt den Besucher über dessen Dasein. Mittels eines Knochenschallgebers kann der Benutzer Informationen über die jeweiligen Kunstobjekte erhalten. Man muss lediglich das Ende des Dojos hinter das Ohr halten und hört die ``Geisterstimme'' mit den Ausführungen zum Objekt. Dojo kommt ohne Kopfhörer aus und ist somit hygienischer als Systeme mit Kopfhörer.
Der Prototyp soll demonstrieren, wie die Technik in das kleine Gehäuse integriert wird, wie das Museumspersonal den Dojo auf den Kunden abstimmt und wie das Gerät während des Betriebs eingesetzt wird.\\
Der Besucher verfügt über die Möglichkeit, Kunstwerke zu ``liken''. Der Dojo merkt sich dies und bietet die Möglichkeit, beim Checkout zusätzliche Informationen zu den ``gelikten'' Objekten (z.B. in Form einer automatisch generierten Broschüre) zu liefern.
Des Weiteren dient der Dojo als intelligentes Eintrittsticket. In einigen Museen gibt es die Möglichkeit, verschiedene Teile einer Ausstellung separat zu besichtigen und dementsprechend nur einen Teil der Eintrittskosten zu zahlen. Solche Zutrittsrechte können auf dem Dojo konfiguriert werden. An einem Zutrittskontrollpunkt hält der Besucher seinen Dojo an einen weiteren Beacon, worauf sich die Schranke bei gewährtem Zutritt öffnet.\\\\
Auf technischer Seite besteht die Herausforderung darin, alle Komponenten in das kleine Gehäuse zu integrieren. Dies ist insbesondere für die Energieversorgung kritisch, da das Gerät mit einer Batterieladung einen Arbeitstag lang versorgt werden muss. Aus diesem Grund wurde beim Lösungskonzept auf hoch energieoptimierte Komponente gesetzt. Die Ansteuerung des Knochenschallgebers erfolgt mit einem sparsamen Klasse D Verstärker. Die Lokalisierung der Beacons sowie die Kommunikation mit Zutrittskontrollpunkten erfolgt mittels Bluetooth Low Energy (BLE) Chips. Diese müssen von der Firmware intelligent gesteuert und ausgewertet werden, da die Lokalisierung von Kunstwerken sowie der Datenaustausch mit Zutrittkontrollpunkten vollautomatisch ablaufen soll.\\\\
Der vorliegende Bericht stellt die technische Dokumentation des Systems dar. Im Kapitel Grundlagen wird der Aufbau und der Funktionsumfang des Produkts geschildert. Danach folgt die technische Lösung der Hardware. Alle Lösungskonzepte und dessen Realisierung werden erläutert und anhand von diversen Messungen validiert. Das Kapitel Software wird auf die gleiche Weise die Lösungsvarianten erläutern und validieren.
Abgeschlossen wird der Bericht durch den Aufbau einer Simulationsumgebung. Diese validiert das Zusammenspiel aller Teilkomponenten als einheitliches Produkt.