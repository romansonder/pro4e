Um das Erlebnis eines Museumsbesuchs zu verbessern, sollten Besucher nicht nur in der Lage seine Ausstellungsstücke anzusehen. Vielmehr sollte es möglich sein, spezifische Informationen auditiv über das Kunstobjekt zu erhalten und diese auf Wunsch mit nach Hause zu nehmen.\\\\
Ziel des Projektes war es, die Technik für einen interaktiven Museumsguide für Museumsbesucher zu realisieren. Das handliche Gerät namens Dojo wurde funktionell und gestalterisch von Jana Kalbermatter entwickelt.  \\
Dojo ist ein portables, stabförmiges Gerät zur Lokalisierung von Kunstobjekten. Es soll umliegende Ausstellungsstücke anhand sogenannter Beacons erkennen und benachrichtigt den Besucher über ihr Dasein. Mittels eines Knochenschallgebers kann der Benutzer Informationen über die jeweiligen Kunstobjekte erhalten. Man muss lediglich die Spitze des Dojos hinter das Ohr halten und hört die ``Geisterstimme'' mit den Ausführungen zum Objekt. Dojo kommt ohne Kopfhörer aus und ist somit hygienischer als Systeme mit Kopfhörer.

Der Besucher soll über die Möglichkeit verfügen, Kunstwerke zu ``liken''. Der Dojo merkt sich dies und bietet die Möglichkeit, beim Checkout die bereits erhaltenen oder beliebig zusätzlichen Informationen zu den ``gelikten'' Objekten (z.\,B. in Form einer automatisch generierten Broschüre) zu liefern.
Des Weiteren dient der Dojo als intelligentes Eintrittsticket. In einigen Museen gibt es die Möglichkeit, verschiedene Teile einer Ausstellung separat zu besichtigen und dementsprechend nur einen Teil der Eintrittskosten zu zahlen. Solche Zutrittsrechte können auf dem Dojo konfiguriert werden. An einem Zutrittskontrollpunkt hält der Besucher seinen Dojo an einen weiteren Beacon, worauf sich die Schranke bei gewährtem Zutritt öffnet.\\\\
Der realisierte Prototyp soll demonstrieren, wie die Technik in das kleine Gehäuse integriert wird, wie das Museumspersonal den Dojo auf den Kunden abstimmt und wie das Gerät während des Betriebs eingesetzt wird.\\
Auf technischer Seite bestand die Herausforderung darin, alle Komponenten in das kleine Gehäuse zu integrieren. Dies ist insbesondere für die Energieversorgung kritisch, da das Gerät mit einer Batterieladung einen Arbeitstag lang versorgt werden muss. Aus diesem Grund wurde beim Lösungskonzept auf hoch energieoptimierte Komponente gesetzt. Die Ansteuerung des Knochenschallgebers erfolgt mit einem sparsamen Klasse D Verstärker. Die Lokalisierung der Beacons sowie die Kommunikation mit Zutrittskontrollpunkten erfolgt mittels Bluetooth Low Energy (BLE) Chips. Diese müssen von der Firmware intelligent gesteuert und ausgewertet werden, da die Lokalisierung von Kunstwerken sowie der Datenaustausch mit Zutrittkontrollpunkten vollautomatisch ablaufen soll.\\\\
Unser Projektteam bestand aus 6 Mitglieder. Dominic Hintbrunner übernahm die Projektleitung. Die Arbeiten wurden in 2 grobe Bereiche eingeteilt. Die Verantwortung der Hardware übernahm Alexander Stutz, und die Verantwortung der Software lag bei unserem Projektleiter Dominic Hintbrunner. Die restlichen Projektmitglieder haben jeweils für beide Projektbereiche gearbeitet und haben für beide Bereiche verschiedene kleinere Teilgebiete übernommen.\\
Die Arbeiten im Projekt 4 hat unser Team als sehr positiv empfunden. Wir sind in unserem Projekt gut vorangekommen, und konnten unser Wunschziel, einen funktionierenden Prototypen im Dojo-Format zu realisieren, verwirklichen. Neben wenigen technischen Schwierigkeiten hatten wir im Team sehr wenige negativen Erfahrungen gemacht.