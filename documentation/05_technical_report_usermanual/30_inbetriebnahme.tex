\textbf{Lithium-Ionen Akku aufladen}
\\[4mm]
Vor der Erstinbetriebnahme sollte der Lithium-Ionen Akku vollständig aufgeladen werden. Dieser Ladevorgang kann bis zu zwei Stunden andauern.
\\[4mm]
Um den Dojo aufzuladen sind folgende Schritte nötig:
\begin{itemize}
\item Computer einschalten.
\end{itemize}
\begin{itemize}
\item USB-Kabel mit USB-Anschluss des Dojos verbinden.
\end{itemize}
\begin{itemize}
\item Das andere Ende des USB-Kabels in einen USB-Port des Computers einstecken.
\end{itemize}
\begin{itemize}
\item Alternativ kann das andere Ende des USB-Kabels auch in einem herkömmlichen 5 Volt USB-Netzteil eingesteckt werden.
\end{itemize}
\textbf{Erstmalige Inbetriebnahme des Dojo}
\\[4mm]
Um Audiodateien von Kunstwerken auf den Dojo zu laden, muss die mitgelieferte Java Testsoftware verwendet werden. Dies hauptsächlich aus dem Grund, da die Audiodateien entsprechend umbenannt werden müssen, um später von der Firmware erkannt werden zu können. Dazu müssen die genannten Schritte in Kapitel \ref{datenaustauschkapitel} auf S.\pageref{datenaustauschkapitel} befolgt werden. Bei Verwendung der Java Testsoftware auf einem Windows Betriebssystem, muss die SD-Karte im Dojo herausgenommen werden und direkt im Computer via SD-Kartenslot verbunden werden.