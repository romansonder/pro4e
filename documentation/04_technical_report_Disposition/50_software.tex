Zuerst wird das Grobkonzept der Software erläutert. Es gibt einen Überblick über alle Teilmodule der Firmware und definiert für jedes Modul die Anforderungen.

\subsection{Kurzeinführung in Bluetooth Low Energy}
Als Einführung werden die wichtigsten Punkte des BLE Standards erläutert. Die verschiedenen Abstraktionsebenen und Rollen eines BLE-Gerätes werden beschrieben.

\subsection{Aufbau des Bluetooth-Moduls}
Jedes Gerät im Projekt 4 hat unterschiedliche Anforderungen an das Bluetooth-Modul. Hier wird beschrieben, welches Geräte welche Funktionen nutzt und wie diese eingesetzt werden.

\subsubsection{Bluetooth-Modul der Peripheriegeräte}
Hier erfolgt (bezogen auf BLE) die ausführliche Beschreibung der Funktionsweise eines Peripheriegerätes.

\subsubsection{Bluetooth-Modul des Dojos}
Dieses Kapitel ist strukturell gleich wie das vorherige, nur bezieht sich der Inhalt auf den Dojo und nicht auf ein Peripheriegerät. Insbesondere sollen folgende Punkte beschrieben werden:
\begin{itemize}
\item Fortlaufende Lokalisierung der Beacons
\item Erkennung von Zugangskontrollpunkten sowie des PC-Dongles
\item Automatisierter Auf- und Abbau einer Verbindung
\item Automatisierter Datenaustausch zwischen den Geräten
\end{itemize}

\subsection{Ein- und Ausgabeelemente des Dojos}
Dieser Abschnitt erklärt die Ansteuerung und Nutzung der übrigen Hardwarekomponenten. Diese sind:
\begin{itemize}
\item Audiochip via SPI
\item Vibrationsmodul via GPIO
\item Buttons und LEDs via GPIO
\item USB-zu-Seriell-Multiplexer
\item SD-Karten-Multiplexer
\end{itemize}
Des Weiteren wird erläutert, wie der Audiochip mit dem Hauptprogramm kommuniziert und wie Statusabfragen gehandhabt werden.

\subsection{Hauptprogrammablauf}
Dieser Abschnitt verknüpft das Zusammenspiel aller Firmware-Module.
Es wird erklärt, wie sich die Firmware in welcher Situation verhält und wie Ereignisse des Bluetooth-Moduls gehandhabt werden.

\subsection{Java Anwendung}
Hier wird kurz erläutert, wozu die Java Anwendung benötigt wird und welche externen Bibliotheken zur Umsetzung verwendet wurden. Eine genaue Beschreibung zur Bedienung (aus sicht des Benutzers) der  Anwendung folgt in einer separaten Bedienungsanleitung.

\subsection{Testumgebung}
Der Abschluss dieses Kapitels wird mit dem Aufbau der Simulationsumgebung gemacht. Die Charakterisierung des Produkts erfolgt durch diverse Messungen und Testversuche. Einerseits werden die Funktionen des Software-Moduls getestet, andererseits das Zusammenspiel des gesamten Systems.