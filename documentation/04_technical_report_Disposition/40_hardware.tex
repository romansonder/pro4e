Hier wird das Grobkonzept der Hardware vorgestellt. Dies gibt einem einen Überblick über die verwendeten Komponenten und dessen Zusammenspiel.

\subsection{Energieversorgung}
Dieser Abschnitt beschreibt die Energieversorgung des Dojos. Dabei sollen die technischen Daten, der Lade- und Entladevorgang sowie der Tiefentladungsschutz beschrieben werden.

\subsubsection{Validierung}
Anstelle eines separaten Abschnitts für die Validierung des Produkts werden die Messungen für die einzelnen Komponenten gleich im dazugehörigen Kapitel gemacht. Hier werden die Messprotokolle über die Lade und Entladekurven aufgeführt. Diese werden benötigt, um die technischen Daten der Energieversorgung zu evaluieren.

\subsection{USB-Schnittstelle}
Hier wird kurz der Konvertierungschip FUP300 und seine Schaltung dokumentiert.

\subsection{Audio-Schnittstelle}
Dieser Abschnitt widmet sich dem Audiochip und der Soundausgabe über den Knochenschallgeber sowie die Treiberstufe zur Ansteuerung.

\subsubsection{Validierung}
Wie zuvor bei der Energieversorgung werden hier die Messungen dieser Teilkomponenten dokumentiert. Es wird eine Impedanzmessung des Knochenschallgebers gemacht und der Frequenzgang der Treiberstufe bestimmt.

\subsection{Zugriff auf SD-Karte}
Dieser Abschnitt widmet sich dem Umschalten der SD-Karte zur USB-Schnittstelle und dem Audiochip. Es wird dabei Wert auf die technischen Daten dieser Verbindung gelegt.

\subsection{Mikrokontroller}
In diesem Abschnitt wird kurz die Wahl des Mikrokontrollers begründet.