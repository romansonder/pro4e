\subsection{Energieversorgung} 
Die Elektronik des Dojos bezieht die Energie von einem Akkumulator. Verwendet wird ein Lithiumakkumulator der Marke Trustfire mit einer Nennspannung von $3.7V$. Der Akkumulator besitzt eine Ladungskapazität von $600mAh$.

\subsubsection{Entladevorgang} 
Der Akkumulator besitzt eine Nennspannung von $4.2V$ bei voller Kapazität und $0V$ \ref{test_Spannungsversorgung} wenn die Kapazität erschöpft ist. Die $0V$ kommen daher zustande, dass der Tiefenentladungsschutz der Batterie die Spannungsversorgung ab einer Spannung unter $2.75V$ kappt. Aus der Batteriespannung wird mit einem Linearregler des Types TL1963A eine $3.3V$ Spannungsversorgung erstellt. Mit dieser Versorgung wird die gesammte Elektronik gespiesen.

\subsubsection{Ladevorgang}
Der Akkumulator wird über die Spannungsversorgung des USB-Ports geladen. Ein Akkumulator-Management-Chip sorgt dabei für eine konstante Spannung von $4.2V$. Diese Spannung wird benötigt um den Akkumulator zu laden. Die Ladezeit beträgt $2.5h$, siehe Messungen Ladevorgang \ref{sec:Ladevorgang}. Somit ist es möglich, den Akkumulator über eine Nacht komplett zu laden, oder ihn zwischen den Benützungen einmal zu laden. 

\subsubsection{Messungen}
Um die Energieversorgung zu verifizieren wurden 3 verschiedene Messungen durchgeführt. Es wurde der Lade- und Entladevorgang aufgezeichnet, sowie die maximale mögliche Last ausgemessen.

\subsubsection*{Maximale Last}
Um herauszufinden welche Leistung bereitgestellt werden kann, wurde die Energieversorgung mit verschiedenen Lasten betrieben. 

\begin{table}[h]
\centering
\label{messungen_Energie}
\begin{tabular}{|l|c|c|c|}
\hline
Last [$\Omega$]     & 100   & 50    & 10    \\ \hline
Batteriestrom [$mA$]    & 32.55 & 63.79 & 293.3 \\ \hline
Batteriespannung [$V$] & 3.823 & 3.735 & 2.979 \\ \hline
Laststrom [$mA$] & 31.06 & 62.03 & 278.1 \\ \hline
Lastspannung [$V$]  & 3.293 & 3.292 & 2.95  \\ \hline
Leistung [$W$]  & 0.124 & 0.238 & 0.874  \\ \hline
\end{tabular}
\caption{Messungen mit verschiedene Widerstände}
\end{table}

Diese Daten zeigen, dass mit einer Leistung von $1W$ die Versorgungsspannung des Printes lediglich auf $2.95V$ sinkt. Dies entspricht einem Spannungsverlust von  $0.35V$, was in der Tolleranz aller elektronischen Bauteile liegt. Da auf unserem Print die maximale Last nur zwischen $0.3 - 0.5W$ liegt, treten keinerlei Probleme auf.
\newpage


\subsubsection*{Entladevorgang}
Um die Batterie zu entladen wurden 2 Messvorgänge durchgeführt. Beim ersten Messvorgang wurde eine Last von $30\Omega$ angeschlossen. Diese Last entspricht einem konstanten Entlagestom von $100mA$, was dem Stromverbrauch unserer Schaltung während dem Abspielen der Musik entspricht. Beim zweiten Messvorgang wurde eine Last von $220\Omega$ angeschlossen, welche einen konstanten Entlagestrom von $15mA$ ergibt. Dieser Entladestrom entspricht dem Verbrauch des Dojos im Ruhestatus. Die Messungen wurde bis zur vollständigen Entladung vollzogen. 
In den Abbildungen \ref{fig:SpannungZuZeit} und  \ref{fig:SpannungZuLadung} ist die Entladekurve für die Spannung im Vergleich zur Zeit und zur Ladung abgebildet:

\begin{figure}[h]
	 \subfigure[Entladekurve bei $100mA$ Entladestrom]{\includegraphics[width=0.49\textwidth]{graphics/SpannungzuZeit}} 
    \subfigure[Entladekurve bei $15mA$ Entladestrom]{\includegraphics[width=0.49\textwidth]{graphics/SpannungzuZeit15}}
	\caption{Entladekurve der Batterie in Funktion der Zeit}
	\label{fig:SpannungZuZeit}
\end{figure}

Wie in der Abbildung \ref{fig:SpannungZuZeit} ersichtlich, hat die Batterie ein Tiefenentladungsschutz bei $2.75V$. Sobald diese Spannung erreicht ist, wird die Spannungsversorgung gekappt, und die Spannung sinkt auf $0V$.\\
Die gesammte Entladung dauert $3h 26min$ bei einem Entladestrom von $100mA$ und $20h50min$ bei einem Entladestrom von 15mA. Die effektive Laufzeit des Dojos befindet sich zwischen diesen Werten. Sie ist abhängig davo, wie stark der Dojo belastet wird, dass heisst, in welchem zyklus neue Audiofiles gestartet werden, und wie lange diese dauern.

\newpage

\begin{figure}[h]
	\centering
	\includegraphics[width=\textwidth]{graphics/SpannungzuLadung.png}
	\caption{Spannung im Verhältnis zur Ladung. Bei einem Widerstand von 30$\Omega$}
	\label{fig:SpannungZuLadung}
\end{figure}

Die Abbildung \ref{fig:SpannungZuLadung} zeigt die abgegebene Ladung im Vergleich zur Batteriespannung. Sie zeigt, dass bei einer Belastung von 100mA und bei einer Belastung von 15mA die Ladung der Batterie im Schnitt $335mAh$ beträgt. Dies entspricht nicht der angegebenen Ladungskapazität von 600mAh, die der Hersteller verspricht. Die Messung zeigt uns, das der Hersteller die Angaben zu optimistisch ermittelt hat. Die Messung zeigt uns auch, dass die benötigte Spannung von $3.3V$ lange gehalten werden kann und erst bei einem Bezug von $320mAh$ die Spannung zusammenfällt.

\newpage

\subsubsection*{Ladevorgang}
\label{sec:Ladevorgang}
Beim Aufladen der Batterie wurde eine Spannung von $5V$ verwendet. Dabei wurde vor dem Anschliessen der Batterie ein Strom von $1.25mA$ gemessen. Die Ladekurve ist in der Abbildung                       \ref{fig:Ladeleistung} ersichtlich.

\begin{figure}[h]
	\centering
	\includegraphics[width=\textwidth]{graphics/ladekurve.png}
	\caption{Ladekurve bei einer konstanten Eingangspannung von 5V}
	\label{fig:Ladeleistung}
\end{figure}

Die gestrichelte Fläche repräsentiert die Verlustleistung der Ladeschaltung beim Laden des Dojos.
Wir erhalten einen Wirkunsggrad von
\begin{equation}
\eta = 83.39%
\end{equation}
Das komplette Aufladen der Batterie dauerte $2.5h$. Somit kann das Museum den Dojo über Nacht problemlos aufladen. Auch Aufladungen zwischen Besuchen können so realisiert werden.
\clearpage

\subsection{USB-Schnittstelle}

\subsection{Audio-Schnittstelle}

\subsubsection{Messungen}

\subsection{Zugriff auf SD-Karte}

\subsection{Mikrokontroller}