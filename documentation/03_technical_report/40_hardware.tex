Um herauszufinden welche Leistung von der Energieversorung bereitgestellt wird, wurden einige Test durchgeführt.

Zuerst wurde die Messung ohne zusätzlichen Widerstand durchgeführt. Somit kann die Leerlaufleistung der Batterie bestimmt werden. Dies führte zu folgenden Werten:
\begin{equation}
I = 1.07mA
\end{equation}

\begin{equation}
U_{Ndl} = 3.295V
\end{equation}

\begin{equation}
U_{B} = 3.925V
\end{equation}
dabei ist $I$ der gemessene Srom, $U_{Ndl}$ ist die Spannung welche nach dem Linearregler gemessen wurde und $U_{B}$ ist die Batteriespannung. Somit beträgt die Leistung der Ladeschaltung im Leerlauf:
\begin{equation}
P_{LL}  = 4.199mW
\end{equation}

Die Messungen mit einem angeschlossenen Widerstand haben ergeben: \\

\begin{table}[hp]
\centering
\label{messungen_Energie}
\begin{tabular}{|l|c|c|c|}
\hline
Widerstand [$\Omega$]     & 100   & 50    & 10    \\ \hline
Batteriestrom [$mA$]    & 32.55 & 63.79 & 293.3 \\ \hline
Batteriespannung [$V$] & 3.823 & 3.735 & 2.979 \\ \hline
Widerstandsstrom [$mA$] & 31.06 & 62.03 & 278.1 \\ \hline
Ausgangsspannung [$V$]  & 3.293 & 3.292 & 2.95  \\ \hline
Leistung [$mW$]  & 0.124 & 0.238 & 0.874  \\ \hline
\end{tabular}
\end{table}