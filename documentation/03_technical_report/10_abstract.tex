To revolutionize museum visits, people should not only be able to look at museum objects. Rather it should be possible to enjoy specified audio information depending on which art objects the visitor is in front of.\\\\ To meet this requirement a device called Dojo was developed. It can recognise Bluetooth Beacons mounted on the art objects and provides the proper audio information via bone conductor. This improves hygiene and makes the use of headphones no longer required. The audio information itself is stored on a built in micro SD card. Further the Dojo can be used for access control purposes to open doors to additional showrooms. To set up visitor specific preferences such as language or access right, a user-friendly computer application was developed. The Dojo simply needs to be in the range of the computers transmitter station to transfer the configurations. With a like-button mounted on the case of Dojo, the visitor can ``like'' art objects to get a personalised summary at the check-out. To ensure fast availability the Dojo is supplied by a 335mAh battery which can be recharged in less than three hours. With this capacity it can operate for at least 3.5 hours. Further up to 500 different audio files can be stored on the 1 GB memory of the Dojo.\\\\ Dojo combines the functionality of a universal museum guide and access authorization in one device. It improves the experience of the visitor and simplifies the handling of an exhibition for the museum staff.\\\\Keywords: museum guide, bone conductor, digital access control
