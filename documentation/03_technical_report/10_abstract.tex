To revolutionize museum visits, people should not only be able to look at museum objects. Rather it should be possible to enjoy specified audio information depending on which art objects the visitor is in front of.\\\\ To meet this requirement a device called the Dojo was developed, which can recognise Bluetooth Beacons mounted on the art objects and providing the proper audio information via bone conductor. The audio information self is stored on a micro SD card in the Dojo. Further the Dojo can be used for access control purposes to open doors to additional art rooms. In addition, the Dojo has a USB interface to charge the battery and to synchronise new art objects with the Dojo-Testsoftware. To transfer visitor specific preferences from the Dojo-Testsoftware like preferred language and access rights, the Dojo simply needs to be in the range of the transmitter station. With a Like-Button mounted on the Dojo casing the visitor can like art objects to get a personalised museum-history at the end of visit.\\\\ The developed Dojo is supplied by a battery with enough capacity to enjoy audio content for three hours. Further up to about 500 different audio contents can be stored on the 1 Gb limited storage of Dojo. At least fast availability of Dojo is ensured with a battery loading time of less than two hours.\\\\ Keywords: museums guide; bone conductor

