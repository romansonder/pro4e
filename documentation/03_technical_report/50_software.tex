\subsection{Audiointerface -- wtv020sd.c}
Das Modul wtv020sd.c beinhaltet die Initialisierung und Ansteuerung des Audiochip Wtv020sd \cite{WTV020}. Die Kommunikation zum Audiointerface erfolgt über die Standard GPIO (General Purpose Input Output) des nRF52. Es werden die Pins: Reset, Busy, Clk und Data verwendet. Über das Audiointerface können die Audiofiles der SD-Karte gezielt Abgespielt, Pausiert und die Audioausgabe vermindert oder verstärkt werden (Mehr Details zum Hardwareaufbau im Kapitel \ref{}). Dabei handelt es sich um 16-Bit Befehle welche über ein SPI (synchroner serieller Datenbus) mit einer Geschwindigkeit von 5kHz gesendet werden.


\subsection{Peripherie -- ts3a27518e.c ts4871.c periphery.c }
Im Modul ts4871.c ist definiert mit welchem Pin der Audioverstärker ein- und ausgeschalten werden kann, in diesem Falle ist es der Standbypin (gem. Datenblat \cite{TS4871}). Des weiteren sind im Modul ts3a27518e.c die Pinausgänge für den Multiplexer, zwischen SD-Karte und Audiointerface definiert (gem. Datenblatt \cite{TS3A27518E}). Es kann zwischen Adio- und USB-Modus gewechselt werden.\\
Das Modul periphery.c beinhaltet die Verarbeitung der Tastereingaben und die Ausgabe des Feedbacks über den Vi1
