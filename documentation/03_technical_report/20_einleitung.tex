Ziel dieses Projektes ist es, ein Museumsführer namens Dojo zu realisieren. Dojo wurde funktionell und gestalterisch von Jana Kalbermatter entwickelt. Dojo ist ein moderner Museumführer, der ohne Kopfhörer auskommt. Mittels Knochenschallgeber soll der Benutzer die Informationen über die jeweiligen Kunstobjekte erhalten. Das stabförmige Gerät informiert via Körperschallübertragung über alle Kunstobjekte in dessen Nähe man sich aufhält. Man muss lediglich das Ende des Stabes hinter das Ohr halten und hört die «Geisterstimme» mit den Ausführungen zum Kunstobjekt. Somit wird kein Kopfhörer benötigt und dadurch ist der Dojo hygienischer als Kopfhörer basierte Museumsführer.
Der Prototyp soll demonstrieren wie das Produkt geladen wird, wie Museumsdaten aktualisiert werden können, wie das Museumspersonal das Dojo auf den Kunden abstimmt und wie das Dojo während des eigentlichen Museumsbesuches eingesetzt werden kann. Dabei soll der Museumbesucher Kunstwerke "Liken" können und gleichzeitig soll ihm mitgeteilt werden, dass er ein Kunstwerk "geliked" hat.
Beim verlassen des Museums erhält man eine persönliche Museums-History. Durch den Dojo wird der Museumsbesuch wieder zum erlebnis.