Ziel dieses Projektes ist es, ein Museumsführer namens Dojo zu realisieren. Dojo wurde funktionell und gestalterisch von Jana Kalbermatter entwickelt. Dojo ist ein moderner Museumführer, der ohne Kopfhörer auskommt. Mittels Knochenschallgeber soll der Benutzer die Informationen über die jeweiligen Kunstobjekte erhalten. Das stabförmige Gerät informiert via Körperschallübertragung über alle Kunstobjekte in dessen Nähe man sich aufhält. Man muss lediglich das Ende des Stabes hinter das Ohr halten und hört die «Geisterstimme» mit den Ausführungen zum Kunstobjekt. Somit wird kein Kopfhörer benötigt und dadurch ist der Dojo hygienischer als Kopfhörer basierte Museumsführer.\\
Der Prototyp soll demonstrieren wie das Produkt geladen wird, wie Museumsdaten aktualisiert werden können, wie das Museumspersonal das Dojo auf den Kunden abstimmt und wie das Dojo während des eigentlichen Museumsbesuches eingesetzt werden kann. Ausserdem soll der Museumbesucher Kunstwerke "Liken" können und gleichzeitig soll ihm mitgeteilt werden, dass er ein Kunstwerk "geliked" hat. Da der Dojo die Museums-Hystorie des Besuchers merkt, erhalten die Besucher beim verlassen des Museums die Möglichkeit ihre "geliketen" Kunstobjekte in eine Broschüre umzuwandeln. \\
In einigen Museen gibt es die Möglichkeit nur ein Teil der Ausstellung zu besichtigen. Dementsprechend zahlt man dafür auch weniger. Der Dojo kann hierbei als virtuelles Ticket dienen. Mittels Sender(Beacon) am jeweiligen Eingang des Ausstelungsabschnittes, erkennt die Türe, ob der jeweilige Dojo die Zutrittsberechtiung hat oder nicht. Möglich wäre auch verschiedene Zutritssberechtigungen zu verleihen, falls das Museum mehr als nur 2 Bereiche hat.
Durch den Dojo braucht man somit nicht mehr für die verschiedenen Bereiche des Museums anzustehen, der  Dojo teilt einem Besucher automatisch mit ob er in dieses Gebiet kann oder nicht.
Somit wird durch den Dojo der Museumsbesuch zum Erlebnis.